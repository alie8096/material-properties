\documentclass[xcolor=dvipsnames,professionalfonts]{beamer}

\usepackage{xcolor}
\usefonttheme{serif}
\usepackage{tikz}
\usepackage{xepersian}
\settextfont{Yas.ttf}

% Uncomment this if you have a custom theme
% \usepackage{mytheme}

\usebackgroundtemplate{\includegraphics[width=\paperwidth, height=\paperheight]{background-brightness-3.jpg}}

\author{علی ابراهیمیان}
\title{پرزنتیشن}
\institute{دانشگاه آزاد اسلامی}

\newtheorem{thm}{قضیه}

\begin{document}

\section{بخش ۱}
\frame{\maketitle}

\begin{frame}
    \frametitle{عنوان اسلاید اول}
    \begin{block}{}
        این اولین صفحه اسلایدر است.
        \begin{itemize}\raggedright
            \item گزینه اول
            \item گزینه دوم
            \item گزینه سوم
        \end{itemize}
    \end{block}
\end{frame}

\begin{frame}
    \frametitle{عنوان اسلاید دوم}
    \begin{block}{}
        این یک فرمول است
        \[
        \int_{0}^{1}{f(x)}dx
        \]
    \end{block}
\end{frame}

\begin{frame}
    \frametitle{وقفه برای رفتن به  متن بعدی}
    \begin{block}{تستی}
        این متن تستی است
        \begin{itemize}
            \item علی
            \item مهدی
            \pause
            \item هادی
            \pause
        \end{itemize}
    \end{block}
\end{frame}

\begin{frame}
    \frametitle{ریاضی}
    \begin{thm}
        فرض کنید
        \begin{itemize}\raggedright
            \item تابع مشتق پذیر باشد.
        \end{itemize}
    \end{thm}
\end{frame}

\end{document}
